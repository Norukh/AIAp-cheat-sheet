%! Author = nico
%! Date = 03.04.23
\documentclass[a4paper, landscape, 10pt]{scrartcl}
\usepackage{csquotes}
\usepackage{subfiles}

% import template
\input{template/template.tex}

% document information
\newcommand{\SUBJECT}{}
\newcommand{\TITLE}{Cheat Sheet für AI Applications (AIAp)}
% Document
\begin{document}

    \begin{multicols*}{3}
        \setlength{\columnseprule}{0pt}
        \footnotesize

        \section{Artificial Neural Network (ANN)}
        \includegraphics[width=\columnwidth]{graphic/03-neural-network}

        \section{Convolutional Neural Network (CNN)}
        \includegraphics[width=\columnwidth]{graphic/02-convolution-architecture}
        Very efficient operation for image processing: convolution.

        \subsection{Advantages over ANN}
        \begin{itemize}
            \item \textbf{Sparse Interaction}
            \subitem Every input element (pixel) is not connected to every neuron
            \subitem Kernel is smaller than input
            \subitem Store fewer parameters, reduces memory requirement of the model
            \item \textbf{Shared Parameters}
            \subitem Important property that allows CNNs to detect spatial patterns
            \subitem
            \subitem
            \subitem
        \end{itemize}
        
        \subsection{Feature Detection}
        \includegraphics[width=\columnwidth]{graphic/00-brain-feature-detectors}

        \subsubsection{Mathematical Model}
        \textbf{Two Inputs}
        \begin{itemize}
            \item Image (width x height for grayscale, or width x height x 3 for RGB)
            \item Filter \textbf{(=kernel)}
            \subitem m by n matrix (1 channel grayscale)
            \subitem m by n x 3 \enquote{stack of matrices} in case of 3 channel input (RGB)
            \subitem m by n x d \enquote{stack of matrices} (depth of kernel must equal number of input channels)
        \end{itemize}
        \textbf{Operation}
        \begin{itemize}
            \item Convolution, input image gets convolved with the convolutional kernel
        \end{itemize}
        \textbf{One Output}
        \begin{itemize}
            \item Feature Map (where is feature active, similar to grayscale image with one channel)
            \item One convolution produces one feature map
        \end{itemize}

        \subsection{Process}
        \includegraphics[width=\columnwidth]{graphic/01-convolution}
        \begin{itemize}
            \item Sliding one kernel over entire input produces one feature map
            \item One kernel combines multiple input channels (e.g. 3 color image, stack of 96 feature maps)
            \item At each layer mutliple different kernels can be applied
            \item From a single input we can produce multiple feature maps
        \end{itemize}

        \subsection{Parameter Stride}

        \subsection{Parameter Padding}

        \subsection{Pooling Variants}
        \subsubsection{MaxPooling}
        \subsubsection{AveragePooling}
        
        \subsection{Softmax Function}

        \subsection{Sparse Categorical Crossentropy Loss}

        \subsection{Regularization Method}
        \subsubsection{DropOut}
        
        \subsection{Limitations of Convolution}

        
        \section{Autoencoder}
        \subsection{Applications}
        \subsubsection{Compression}
        \subsubsection{Pretraining useful Conv Kernels without Labels}
        \subsubsection{Denoising}

        \section{Optimization}
        \subsection{Data Augmentation}
        \subsection{Learning Curves}
        \subsubsection{Underfitting}
        \subsubsection{Overfitting}

        \subsection{Train, Validation, Test}
        \begin{itemize}
            \item Training Set
            \subitem Used to optimize model
            \item Validaton Set
            \subitem Used to evaluate model performance during training
            \subitem
        \end{itemize}

        \subsection{k-Fold Cross Validation}


        \section{Recurrent Neural Network (RNN)}
        \subsection{Applications}
        \begin{itemize}
            \item Stock prices forecasting
            \item Weather prediction
            \item Forecasting traffic
            \item Text generation
            \item Anomaly Detection
        \end{itemize}

        \subsection{Recurrence}
        
        \subsection{Memory Cell}

        \subsection{SimpleRNN}
        

        \subsection{Training}
        \subsubsection{Backpropagation}
        \begin{enumerate}
            \item Initialize Weights
            \item Until Convergence
            \subitem Forward pass through network for set of inputs
            \subitem Calculate Loss
            \subitem Calculate Gradient
            \subitem Update Weights
        \end{enumerate}
        
        \subsection{Limitations of vanilla RNNs}
        \subsubsection{Exploding Gradient Problem}


        \subsubsection{Vanishing Gradient Problem}

        \subsubsection{Solutions}
        \begin{itemize}
            \item Activation Functions
        \end{itemize}

        \subsection{DeepRNN}

        \subsection{Bidirectional RNN}

        \subsection{LSTM \tiny{Long-Short-Term-Memory}}


        \subsubsection{Gates}
        \begin{enumerate}
            \item Forget
            \item Store / Compute / Input
            \item Cell / Update
            \item Output
        \end{enumerate}

        \subsubsection{Forget Gate}
        \subsubsection{Input Gate}
        \subsubsection{Update Gate}
        \subsubsection{Output Gate}
        
        \subsection{GRU \tiny{Gated Recurrent Units}}
        \subsubsection{GRU Gates}
        \begin{itemize}
            \item Reset Gate
            \item Update Gate
        \end{itemize}

        \subsubsection{Reset Gate}
        \subsubsection{Update Gate}

        \subsubsection{Output}

        \subsection{LSTM or GRU?}

        \begin{tabularx}{\columnwidth}{X | X}
            \textbf{LSTM} & \textbf{GRU} \\
            + gradient descent more variability & + faster, compacter, fewer parameters, faster training \\
            \hline
            + large datasets & + smaller datasets \\
            \hline
            + longer sequence & \\
            + better suited for modelling long-range correlations & \\
            \hline
            + 3 Gates (forget, input, output) & + 2 Gates (reset, update) \\
            \hline
            & + controls flow of information without cell memory unit \\
            \hline
            & + expose complete memory without any control
        \end{tabularx}

        TODO: continue with ATTENTION (WEEK 7)

    \end{multicols*}

\end{document}